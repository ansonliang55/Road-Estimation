%%%%%%%%%%%%%%%%%%%%%%%%%%%%%%%%%%%%%%%%%%%%%%%%%%%%%%%%%%%%%%%%%%%%%%
% LaTeX Example: Project Report
%
% Source: http://www.howtotex.com
%
% Feel free to distribute this example, but please keep the referral
% to howtotex.com
% Date: March 2011 
% 
%%%%%%%%%%%%%%%%%%%%%%%%%%%%%%%%%%%%%%%%%%%%%%%%%%%%%%%%%%%%%%%%%%%%%%
% How to use writeLaTeX: 
%
% You edit the source code here on the left, and the preview on the
% right shows you the result within a few seconds.
%
% Bookmark this page and share the URL with your co-authors. They can
% edit at the same time!
%
% You can upload figures, bibliographies, custom classes and
% styles using the files menu.
%
% If you're new to LaTeX, the wikibook is a great place to start:
% http://en.wikibooks.org/wiki/LaTeX
%
%%%%%%%%%%%%%%%%%%%%%%%%%%%%%%%%%%%%%%%%%%%%%%%%%%%%%%%%%%%%%%%%%%%%%%
% Edit the title below to update the display in My Documents
%\title{Project Report}
%
%%% Preamble
\documentclass[paper=letter, fontsize=11pt]{scrartcl}
\usepackage[T1]{fontenc}
\usepackage{fourier}

\usepackage[english]{babel}															% English language/hyphenation
\usepackage[protrusion=true,expansion=true]{microtype}	
\usepackage{amsmath,amsfonts,amsthm} % Math packages
\usepackage[pdftex]{graphicx}	
\usepackage{url}
\usepackage{float}
\usepackage{ mathrsfs }
\restylefloat{table}
\usepackage[shortlabels]{enumitem}
%%% Custom sectioning
\usepackage{sectsty}
\allsectionsfont{\centering \normalfont\scshape}


%%% Custom headers/footers (fancyhdr package)
\usepackage{fancyhdr}
\pagestyle{fancyplain}
\fancyhead{}											% No page header
\fancyfoot[L]{}											% Empty 
\fancyfoot[C]{}											% Empty
\fancyfoot[R]{\thepage}									% Pagenumbering
\renewcommand{\headrulewidth}{0pt}			% Remove header underlines
\renewcommand{\footrulewidth}{0pt}				% Remove footer underlines
\setlength{\headheight}{13.6pt}


%%% Equation and float numbering
\numberwithin{equation}{section}		% Equationnumbering: section.eq#
\numberwithin{figure}{section}			% Figurenumbering: section.fig#
\numberwithin{table}{section}				% Tablenumbering: section.tab#


%%% Maketitle metadata
\newcommand{\horrule}[1]{\rule{\linewidth}{#1}} 	% Horizontal rule

\title{
		%\vspace{-1in} 	
		\usefont{OT1}{bch}{b}{n}
		\normalfont \normalsize \textsc{UoT CSC2515 - Introduction to Machine Learning Spring 2015} \\ [25pt]
		\horrule{0.5pt} \\[0.4cm]
		\huge Evaluation of Various Machine Learning Techniques for Road-Estimation \\
		\horrule{2pt} \\[0.5cm]
}
\author{
		\normalfont 								\normalsize
        Jia Ming Liang, Wenjie Zi\\[-3pt]		\normalsize
        \{ansonlia, wenjie\}@cs.toronto.edu\\[-3pt]		\normalsize
        \today
}
\date{}


%%% Begin document
\begin{document}
\maketitle
\section{Introduction}

One problem with autonomous driving car is to recognize the road in the scene captured by a camera setup in the front of the car. This is a binary classification problem because we want to determine whether a pixel in the captured scene belongs to road or non-road. The goal of our project is to evaluate different machine learning techniques that can be used to solve this problem. \\

For the purpose of our project, we use the base kit of KITTI with training label data \cite{bib:kitti}. The base kit contains 289 training images and 289 labeled images (ground truth image labeling the road). There are 3 types of scene in these training images: 95 urban marked (UM) images, 98 urban unmarked images and 96 urban multiple marked lanes images. "Marked" means that the road has lane marking. We split the training data randomly into training (60\%), validation (10\%) and testing (30\%). This allows us to use the k-fold cross-validation technique.\\


Our program is mainly written in Python. The core parts of our code include: feature extraction, data training and results evaluation. To simplify some of our implementation, we have utilized various Python libraries in our code including skimage and sklearn \cite{bib:skimage,bib:sklearn}. We will explain how we use these libraries later in this report.

\if
BONUS: The use of "fancy" techniques such as Markov Random Fields should only be attempted if you want an extra bonus on the project (up to 30\% extra). These type of techniques are only cover in class as a review of exciting things in machine learning and thus I do not expect students to try such things for the purpose of the project.
\fi
\section{Feature Extraction}
The main part of our solution to the road-estimation problem is extracting features from the training images. In this section, we will explain what we implemented to extract the input image features and how each feature affect our result. 

\subsection{Superpixels}
Our input image has the size 1242x375, which gives us 465750 pixels per image. Since we have 289 training images, that give us about 135 million pixels to process. To easy the computation, we first compute the superpixel for each input image using the SLIC superpixel algorithm \cite{bib:slic}. For simplification purposes, we used the slic function in the skimage library \cite{bib:skimage} to directly compute the superpixel for each image.

<need to finsih this using some image explaination with various parameter set and results>

\subsection{Basic Features}
Now our classification is done on superpixels, our features need to be extracted from these superpixels. The most obvious feature for the superpixels feature is color. Based on the SLIC superpixel results, we can see that very similar color neighboring pixels are grouped together to form one superpixel. Hence, we first use the mean color of the superpixel by averaging out each of the red, green and blue color for all the pixels within a superpixel. The resulting image looks like Figure \cite{fig:meanslic}. \\

Since the training images are taking by a camera in front of the car facing the road, the road is almost always located at the center bottom of the image. Pixels are less likelily to be road the further away they are from the center bottom of the image. Of course, this applies to superpixel as well. Hence, the location of the superpixel is also an important feature.\\ 

Also, most larger superpixels are located at textures that look very flat and have very little edges. Although superpixels that are located at textures that are flat and non-road are also larger but at least we can eliminate the ones that are smaller and close to the edge of the road (see Figure \cite{fig:smallsuperpixel}). Hence, counting number of pixels in a superpixel can be used as a feature.\\

We also observe that the superpixel that have a irregular shape  are usually not part of the road. Hence, we can extract the shape of the superpixel as the feature as well. We can simply use the width and length of the superpixel.\\

Now we have described all the basic features we extracted from a superpixels:
\begin{enumerate}
	\item 3 dimensions superpixel mean color (R, G and B).
	\item 2 dimensions superpixel location (x and y).
	\item 1 dimension superpixel size (pixels count)
	\item 2 dimensions superpixel shape (height and width)
\end{enumerate}

In total we have 8 dimensions of basic feature vector. In the results evaluation section, we will discuss how each feature affect the accuracy of the prediction.





\subsection{Histogram of Gradient}
\subsection{Color Histogram}
\subsection{Scaling to pixel-wise}
Make a section of your report about what would you do to scale what you have done to work with pixels, as in that case you have millions of examples. You don't need to implement anything
but add a discussion of your ideas. 

\section{Data Training}
\subsection{Preprocessing}
\subsection{Naive Bayes}
\subsection{K-nearest Neighbor}
\subsection{Adaboost}
\subsection{Random Forest}
\subsection{Support Vector Machine}

\section{Results Evaluation}

\section{Conclusion}

\begin{thebibliography}{9}

\bibitem{bib:kitti}
  http://www.cvlibs.net/datasets/kitti/eval$\_$road.php
\bibitem{bib:slic}
  http://ivrg.epfl.ch/research/superpixels$\#$SLICO. 
\bibitem{bib:skimage}
  http://scikit-image.org/docs/dev/api/skimage.segmentation.html$\#$skimage.segmentation
\bibitem{bib:sklearn}
  http://scikit-learn.org/stable/
\end{thebibliography}




\if

Below are my results for running all 4 images
\begin{figure}[H]
\begin{center}
\includegraphics[width=100mm]{result1.png}
\end{center}
\caption{Results set cellGFP2a\_00}
\label{fig:resultset1}
\end{figure}
\begin{figure}[H]
\begin{center}
\includegraphics[width=100mm]{result2.png}
\end{center}
\caption{Results set cellGFP2a\_10}
\label{fig:resultset2}
\end{figure}
\begin{figure}[H]
\begin{center}
\includegraphics[width=100mm]{result3.png}
\end{center}
\caption{Results set cellGFP2a\_01}
\label{fig:resultset3}
\end{figure}
\begin{figure}[H]
\begin{center}
\includegraphics[width=100mm]{result4.png}
\end{center}
\caption{Results set cellGFP2a\_11}
\label{fig:resultset4}
\end{figure}

\fi
%%% End document
\end{document}